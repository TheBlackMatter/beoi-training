\documentclass[12pt]{beamer}

\usepackage[utf8]{inputenc}
\usepackage{listings,tabu,tikz,amsmath,xcolor,textcomp}
\usepackage[frenchb]{babel}

\definecolor{myblue}{rgb}{0,0,0.7}
\definecolor{mygreen}{rgb}{0,0.5,0}
\definecolor{mygray}{rgb}{0.4,0.4,0.4}
\definecolor{mymauve}{rgb}{0.3,0,0.5}

\lstset{
    language=C++, basicstyle=\footnotesize, frame=single,
    literate=%
        {à}{{\`a}}1
        {â}{{\^a}}1
        {É}{{\'E}}1 {é}{{\'e}}1
        {è}{{\`e}}1
        {ê}{{\^e}}1
        {î}{{\^i}}1,
    commentstyle=\color{mygreen},
    keywordstyle=\bf\color{myblue},
    stringstyle=\color{mymauve},
    showstringspaces=false
}

\beamertemplatenavigationsymbolsempty
\AtBeginSection[]
{
    \begin{frame}
    \frametitle{Table des matières}
    \tableofcontents[currentsection]
    \end{frame}
}

\title{Crash course de C++}
\subtitle{Pour programmeurs en n'importe quel langage raisonnable}
\author{beOI Training}
\institute{\includegraphics[height=12em]{../share/beoi-logo}}
\date{}

\begin{document}

\maketitle


\section{Les bases}

\begin{frame}[fragile]
\frametitle{Hello world!}
\begin{lstlisting}
// Ceci est un commentaire.
#include <bits/stdc++.h> // importe toute la STL
using namespace std; // évite de taper std::

// Déclaration de fonction : type nom() { [...] }
int main() {
    // Toutes les instructions terminent par ";"
    cout << "Hello World!" << endl;
    // endl = retour à la ligne
}
\end{lstlisting}
\begin{itemize}
\item La fonction \lstinline|int main()| est appelée au lancement.
\item Le \lstinline|int| veut dire que \lstinline|main()| renvoie un entier, mais \lstinline|main()| renvoie \lstinline|0| automatiquement (raisons historiques).
\item Le sens des chevrons \lstinline|<<| indique que les mots vont vers \lstinline|cout|, la "sortie standard".
\end{itemize}
\end{frame}

\begin{frame}[fragile]
\frametitle{Compilation et lancement}
Les programmes en C++ doivent d'abord être transformés dans un langage compréhensible par l'ordinateur.

~

\textbf{Compilation} (en ligne de commande)
\begin{lstlisting}
g++ -std=c++11 -Wall -Wextra -Wshadow hello.cpp
\end{lstlisting}
\begin{itemize}
\item \lstinline|g++|: nom du compilateur
\item \lstinline|-std=c++11|: version de C++ utilisée
\item \lstinline|-Wall -Wextra -Wshadow|: active plein de warnings très utiles pour écrire des programmes sans bugs
\item \lstinline|hello.cpp|: code source
\end{itemize}

\textbf{Lancement}: \lstinline|./a.out|
\begin{itemize}
\item \lstinline|./|: dossier dans lequel on se trouve
\item \lstinline|a.out|: nom du fichier exécutable produit par \lstinline|g++|.
\end{itemize}
\end{frame}

\begin{frame}[fragile]
\frametitle{Variables et opérations}
Une variable doit avoir un type spécifié
\begin{lstlisting}
int i = 5; // entiers dans [-2^31, 2^31[
double j = 5.4; // nombres à virgules
bool b = true; // booléen (vrai ou faux)
char ch = 'D'; // caractères seuls
string s = "abcd"; // chaîne de caractères

// On peut les initialiser plus tard
int k,l; // plusieurs variables du même type

k = i + 2;
l = 7 / 3; // division entière => l = 2
s += ch; // ajout d'un caractère
j /= 3; // divise j par 3
i++; l--; // ajoute 1, enlève 1
\end{lstlisting}
\end{frame}

\begin{frame}[fragile]
\frametitle{Conditions}
\begin{lstlisting}
int age;
cout << "Quel est votre âge ? ";
cin >> age; // lecture d'input
if (age < 18)
    cout << "Vous êtes mineur." << endl;
else if (age <= 120)
    cout << "Vous êtes majeur." << endl;
else {
    int a=3, b=4, c=5;
    bool rectangle = (a*a + b*b == c*c);
    if (rectangle && !(a == 0 || b == 0))
        cout << "Hypoténuse = " << c << endl;
}
\end{lstlisting}
\begin{itemize}
\item Les accolades \lstinline|{}| sont facultatives pour une seul ligne.
\item Le sens des chevrons \lstinline|>>| indique que l'entier vient de \lstinline|cin|, "l'entrée standard".
\end{itemize}
\end{frame}

\begin{frame}[fragile]
\frametitle{Boucles}
\begin{lstlisting}
// Imprime les nombres de 1 à 5
for (int i=1; i<=5; i++)
    cout << i << endl;

string s; // initialement vide
while (s != "oui") { // pas égal
    cout << "Aimez-vous programmer ? ";
    cin >> s;
}
\end{lstlisting}
\begin{itemize}
\item Les boucles \lstinline|for(;;)| ont trois parties:
\begin{itemize}
\item Initialisation: initialise une ou plusieurs variables
\item Condition: la boucle s'arrête quand elle est fausse
\item Incrémentation: exécutée à la fin de chaque itération
\end{itemize}
\end{itemize}
\end{frame}

\begin{frame}[fragile]
\frametitle{Fonctions}
\begin{lstlisting}
// Type obligatoire pour résultat et paramètres
int square(int x) {
    return x*x;
}
void sayHello(string s) { // void = ne renvoie rien
    cout << "Bonjour " << s << endl;
}
int main() {
    int y = square(4); // y = 16
    sayHello("Victor");
}
\end{lstlisting}
\begin{itemize}
\item Pas imbriquables, et toujours placées \emph{avant} leur appel. Sinon il faut les déclarer ainsi: \lstinline|void sayHello(string s);| et les implémenter après.
\item Quand une fonction n'est pas \lstinline|void|, toutes les exécutions possibles doivent terminer avec un \lstinline|return|.
\end{itemize}
\end{frame}

\begin{frame}[fragile]
\frametitle{Tableaux}
Toutes les cases d'un tableau doivent avoir le même type.
\begin{lstlisting}
int maxi(int tab[], int n) { // le [] est toujours
    int ma = 0;              // après le nom
    for (int i=0; i<n; i++)
        ma = max(ma, tab[i]);
    return ma; // renvoie le maximum de a
}
int main() {
    int a[5], b[4][3]; // 4 lignes et 3 colonnes
    for (int i=0; i<5; i++)
        cin >> a[i];
    cout << maxi(a, 5) << endl;
}
\end{lstlisting}
\begin{itemize}
\item Le premier élément est à l'indice \lstinline|[0]|.
\item La taille ne peut pas être modifiée.
\item Un tableau ne connaît pas sa taille ! Il faut la donner à part quand on l'envoie à une fonction.
\end{itemize}
\end{frame}


\section{La librairie standard}

\begin{frame}[fragile]
\frametitle{STL et conteneurs}
La librairie standard (STL) contient un tas de structures et fonctions très utiles.

~

Une structure très utile est le \lstinline|vector<>|:
\begin{lstlisting}
vector<int> v(3,-1); // 3 éléments initialisés à -1
v.push_back(7); // ajoute 7 au vecteur
v[1] = 5; // accès comme un tableau
for (int i : v) // prend les éléments un par un
    cout << i << endl; // imprime -1, 5, -1, 7
\end{lstlisting}
\begin{itemize}
\item On met le type des éléments dans les chevrons: \lstinline|<int>|.
\item Les structures ont des constructeurs qui les initialisent (ici \lstinline|(3,-1)|) et beaucoup de méthodes (ici \lstinline|.push_back()|).
\item Plus d'infos sur les conteneurs: \url{http://en.cppreference.com/w/cpp/container}
\end{itemize}
\end{frame}

\begin{frame}[fragile]
\frametitle{Algorithmes STL}
La STL contient aussi beaucoup d'algorithmes prêts à l'emploi:
\begin{lstlisting}
vector<int> v{4,-1,3,2}; // initialisé avec liste
sort(v.begin(), v.end()); // trie tout le vecteur
swap(v[0], v[1]); // échange les contenus
reverse(v.begin(), v.end()); // renverse le vecteur

for (int i : v)
    cout << v << endl; // imprime 4, 3, -1, 2
\end{lstlisting}
\begin{itemize}
\item Et bien d'autres: copies, recherche binaire, matching, sélection du $i$-ème plus petit élément, ...
\item Plus d'infos sur les algorithmes: \url{http://en.cppreference.com/w/cpp/algorithm}
\end{itemize}
\end{frame}


\section{Pointeurs et références}

\begin{frame}
\frametitle{Disclaimer}
Attention, à partir de maintenant ça devient hard.

~

Âmes sensibles s'abstenir!
\end{frame}

\begin{frame}[fragile]
\frametitle{Mémoire et adresses}
La mémoire d'un PC est comme un grand tableau, rempli de cases pour stocker les variables.
\vspace{6pt}
\begin{columns}[T]
\begin{column}{.45\textwidth}
\textbf{Programme}
\begin{lstlisting}
...
int a = 2, b = 3;
char ch[]{'h','e','y'};
...
\end{lstlisting}
\end{column}
\begin{column}{.45\textwidth}
\textbf{Mémoire RAM}
\vspace{4pt}

\begin{tabu} {r|c|c|c|c|}
$\cdots$ & \multicolumn{4}{c|}{$\cdots$} \\
\cline{2-5}
\lstinline|mem[12]| & \multicolumn{4}{c|}{\lstinline|a = 2|} \\
\cline{2-5}
\lstinline|mem[8]| & \multicolumn{4}{c|}{\lstinline|b = 3|} \\
\cline{2-5}
\lstinline|mem[4]| & \lstinline|'h'| & \lstinline|'e'| & \lstinline|'y'| & \\
\cline{2-5}
$\cdots$ & \multicolumn{4}{c|}{$\cdots$}
\end{tabu}
\end{column}
\end{columns}
\vspace{4pt}

\begin{itemize}
\item Chaque case peut contenir un byte (8 bits), une variable peut être étalée sur plusieurs cases (\lstinline|int| 4, \lstinline|double| 8)
\item Chaque case a une \emph{adresse} (\lstinline|4|, \lstinline|8|, \lstinline|12|, ...).
\item L'adresse permet d'accéder à la variable
\end{itemize}
\end{frame}

\begin{frame}[fragile]
\frametitle{Pointeurs}
\begin{itemize}
\item Un pointeur est une variable qui contient l'adresse d'une autre variable.
\item On peut lire ou modifier une variable grâce à son pointeur.
\end{itemize}
\vspace{6pt}
\begin{columns}[T]
\begin{column}{.6\textwidth}
\textbf{Programme}
\begin{lstlisting}
int a = 2;
int *b = &a; // type = (int *)
cout << b << endl;  // 12
cout << *b << endl; // 2
*b = 5; // modifie a, pas b
cout << a << endl;  // 5
\end{lstlisting}
\end{column}
\begin{column}{.3\textwidth}
\textbf{Mémoire RAM}
\vspace{7pt}

\begin{tabu}{r|c|}
$\cdots$ & $\cdots$ \\
\cline{2-2}
\lstinline|mem[12]| & \lstinline|a = 2| \\
\cline{2-2}
\lstinline|mem[8]| & \lstinline|b = 12| \\
\cline{2-2}
$\cdots$ & $\cdots$
\end{tabu}
\end{column}
\end{columns}
\vspace{3pt}
\begin{itemize}
\item \lstinline|&a| = ``indice de \lstinline|a| dans \lstinline|mem[]|''
\item \lstinline|*b| = \lstinline|mem[b]| (et donc on peut lire et écrire à cet indice)
\end{itemize}
\end{frame}

\begin{frame}[fragile]
\frametitle{Passage par pointeur}
\begin{itemize}
\item Quand on envoie une variable à une fonction, elle est copiée (passée \emph{par valeur}).
\item \emph{Si on la modifie, ça ne change pas la variable de base.}
\end{itemize}
\vspace{3pt}
\begin{columns}[T]
\begin{column}{.45\textwidth}
\textbf{Passage par valeur}
\begin{lstlisting}
void add3(int a) {
    a += 3;
}   // a = copie de i
int main() {
    int i=2;
    add3(i);
    cout << i << endl;
}   // affiche 2
\end{lstlisting}
\end{column}
\begin{column}{.45\textwidth}
\textbf{Passage par pointeur}
\begin{lstlisting}
void add3(int *p) {
    *p += 3;
}   // "mem[p] += 3"
int main() {
    int i=2;
    add3(&i);
    cout << i << endl;
}   // affiche 5
\end{lstlisting}
\end{column}
\end{columns}
\vspace{3pt}
À droite, l'adresse \lstinline|p| est aussi passée par valeur, \emph{mais on modifie la valeur de \lstinline|i| directement dans la mémoire avec \lstinline|*p|}.
\end{frame}

\begin{frame}[fragile]
\frametitle{Références}
Les références permettent d'utiliser un pointeur comme une variable normale (sans écrire \lstinline|&| et \lstinline|*| tout le temps).
\begin{lstlisting}
int a = 2;
int &b = a; // type = (int &)
cout << b << endl;  // 2
b = 5; // modifie a à travers b
cout << a << ' ' << b << endl;  // 5 5
\end{lstlisting}
Différences avec les pointeurs:
\begin{itemize}
\item Une référence est liée à une seule variable, pour toujours.
\item \lstinline|b| est comme un autre nom pour \lstinline|a|, c'est un alias.
\item On n'a pas accès à l'adresse de \lstinline|a|.
\item Le \lstinline|&| des références n'a \textbf{rien à voir} avec le \lstinline|&| des pointeurs.
\end{itemize}
\end{frame}

\begin{frame}[fragile]
\frametitle{Passage par référence}
Quand on passe une variable \emph{par référence}, elle n'est pas copiée, c'est la ``même variable''.
\vspace{3pt}
\begin{columns}[T]
\begin{column}{.45\textwidth}
\textbf{Passage par valeur}
\begin{lstlisting}
void add3(int a) {
    a += 3;
}   // a = copie de i
int main() {
    int i=2;
    add3(i);
    cout << i << endl;
}   // affiche 2
\end{lstlisting}
\end{column}
\begin{column}{.45\textwidth}
\textbf{Passage par référence}
\begin{lstlisting}
void add3(int &a) {
    a += 3;
}   // a = alias de i
int main() {
    int i=2;
    add3(i);
    cout << i << endl;
}   // affiche 5
\end{lstlisting}
\end{column}
\end{columns}
\vspace{3pt}
\begin{itemize}
\item La seule différence c'est le \lstinline|&| devant le nom de la variable.
\item À droite, vu que \lstinline|i| et \lstinline|a| sont deux noms pour une même variable, quand on modifie \lstinline|a|, ça modifie \lstinline|i| aussi.
\item Ça évite de copier la variable $\Rightarrow$ plus rapide.
\end{itemize}
\end{frame}

\end{document}
